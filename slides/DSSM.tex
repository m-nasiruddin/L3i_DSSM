 \documentclass{beamer}

\usepackage[british]{babel}
\usepackage{graphicx,hyperref,ru,url}

% The title of the presentation:
%  - first a short version which is visible at the bottom of each slide;
%  - second the full title shown on the title slide;
\title[Laboratoire Informatique, Image et Interaction (L3i)]{Deep Structure Similarity Model (DSSM)}

% Optional: a subtitle to be dispalyed on the title slide
\subtitle{for information retrieval}

% The author(s) of the presentation:
%  - again first a short version to be displayed at the bottom;
%  - next the full list of authors, which may include contact information;
\author[Mohammad Nasiruddin]{Mohammad Nasiruddin}
  % {\small \url{mohammad.nasiruddin@laposte.net}} % \\ 
  % {\small \url{http://www.cs.ru.nl/~pim/}}}

% The institute:
%  - to start the name of the university as displayed on the top of each slide
%    this can be adjusted such that you can also create a Dutch version
%  - next the institute information as displayed on the title slide
\institute[Univ. de La Rochelle]{
  Laboratoire Informatique, Image et Interaction (L3i) \\
  Univ. de La Rochelle}

% Add a date and possibly the name of the event to the slides
%  - again first a short version to be shown at the bottom of each slide
%  - second the full date and event name for the title slide
\date[November 2017]{November 2017, La Rochelle}

\begin{document}

\begin{frame}
  \titlepage
\end{frame}

\begin{frame}
  \frametitle{Outline}

  \tableofcontents
\end{frame}

% Section titles are shown in at the top of the slides with the current section 
% highlighted. Note that the number of sections determines the size of the top 
% bar, and hence the university name and logo. If you do not add any sections 
% they will not be visible.


\section{Introduction}

\begin{frame}
  \frametitle{Introduction}
\end{frame}


\section{PHOCNet architecture}

\begin{frame}
  \frametitle{PHOCNet overview}
\end{frame}


\begin{frame}
  \frametitle{State of the art}
\end{frame}


\begin{frame}
  \frametitle{A deep CNN architecture}
\end{frame}


\begin{frame}
  \frametitle{PHOCNet architecture}
\end{frame}


\section{Experiments and results}

\begin{frame}
  \frametitle{Datasets}
\end{frame}


\begin{frame}
  \frametitle{Results}
\end{frame}


\section{Conclusion}

\begin{frame}
  \frametitle{Conclusion}
\end{frame}


\begin{frame}[allowframebreaks]

  \begin{center}
    \textbf{Thank you!}
  \end{center}
  \begin{center}
    \textbf{Questions?}
  \end{center}
\end{frame}


\begin{frame}[allowframebreaks,noframenumbering]
  \frametitle{Reference}
  
  \tiny
  \bibliographystyle{apalike}
  \bibliography{bibliography.bib}
\end{frame}

\end{document}
